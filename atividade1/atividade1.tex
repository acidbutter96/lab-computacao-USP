\documentclass[18pt]{article}
\usepackage[portuguese]{babel}
\usepackage{mathbbol,amsmath,mathptm,mathrsfs,esint,bm}
\usepackage{amsthm}
\usepackage[T1]{fontenc}
\usepackage[utf8]{inputenc}
%\usepackage[utopia]{mathdesign}
\usepackage{amsfonts}
\usepackage{graphicx,color}
\usepackage[top=3cm, bottom=4cm, left=1.5cm, right=2cm]{geometry}
\usepackage{sidecap}
\usepackage{wrapfig}
\usepackage{hyperref}
\usepackage{cancel}
\usepackage[T1]{fontenc}
\usepackage{lmodern}
\usepackage{amsfonts}
\usepackage{graphicx,color,kantlipsum}
\usepackage{xcolor}
\usepackage[small,it]{caption}
\DeclareMathOperator{\arcsen}{arcsen}
\DeclareMathOperator{\sen}{sen}
\DeclareMathOperator{\senh}{senh}
\DeclareMathOperator{\I}{i}
\DeclareMathOperator{\e}{e}
\DeclareMathOperator{\Div}{div}
%\numberwithin{equation}{subsection} %IMPORTANTE
\usepackage{caption}
\DeclareCaptionType{eqsub}[][List of equations]
\captionsetup[eqsub]{labelformat=empty}
%plot
\newtheorem{defi}{Definição}
\newtheorem{teo}{Teorema}
\newtheorem{demo}{Demonstração}
\usepackage{braket}
\usepackage{multicol}
\usepackage{siunitx}
\newcommand{\laplace}[1]{\mathcal{L}\left\{#1\right\}}
\newcommand{\laplaceinv}[1]{\mathcal{L}^{-1}\left\{#1\right\}}
\newcommand{\dx}{\mathrm{d}x}
\newcommand{\dd}[1]{\hspace{2pt}d#1}
\newcommand{\abs}[1]{\left|#1\right|}
\newcommand{\vers}[1]{\hat{\mathbf{#1}}}
%\newcommand{\dx}{\mathrm{d}x}
%\newcommand{\dd}[1]{\hspace{2pt}d#1}
\title{Agulha de Buffon - Atividade 1}
\author{Marcos Pereira - 11703832}

\usepackage{makecell}

\renewcommand\theadalign{bc}
\renewcommand\theadfont{\bfseries}
\renewcommand\theadgape{\Gape[4pt]}
\renewcommand\cellgape{\Gape[4pt]}



\begin{document}
	\begin{wrapfigure}{r}{0.1\textwidth}
		\centering
		\includegraphics[width=0.19\textwidth]{usp}
		\textsl{}	\end{wrapfigure}
	{\large Agulha de Buffon}
	\\
	
	
	\begin{flushleft}
		Universidade de São Paulo - USP
		\\
		
		Marcos Pereira \hspace{2cm} Nº USP: 11703832\\
		
		Departamento de Matemática Aplicada - MAP
		\vspace{.0cm}
		{\large \textbf{}}
	\end{flushleft}
\vspace{1cm}

\begin{center}
	{\Large \textbf{Agulha de Buffon}} 
\end{center}

\vspace{1cm}

\section{Introdução}

Um dos métodos existentes na Matemática para estimar o valor de $\pi$ é o método da \textbf{agulha de Buffon}, ele foi proposto no século XVII pelo francês Georges-Louis Leclerc, também conhecido como Georges de Buffon ou conde de Buffon.

O método consiste em ....\\

\paragraph{conde de Buffon}

Georges-Louis Leclerc nasceu em Montbard, na França em 7 de setembro de 1707, foi matemático, filosofo e escritor cujas teorias influenciaram duas gerações naturalistas, nas quais faziam partes nomes influentes como Charles Darwin.\\

Georges-Louis Leclerc faleceu em 16 de abril de 1788 aos 80 anos.

\paragraph{Método}

O problema da agulha de Buffon pode ser formulado como se segue: Suponha que tenhamos um piso composto de tiras paralelas de madeira com mesma largura, então jogamos uma agulho no chão. Qual é a probabilidade de que a agulha caia entre duas tiras de madeira?

Esse problema pode ser considerado o primeiro problema na geometria probabilística, a solução para o problema (a probabilidade desejada) para o caso de uma agulha cujo tamanho seja menor do que a largura das tábuas $\ell<L$ é dada por
\begin{align}
P=\frac{2\ell}{\pi L}\,.
\end{align}

Seja $x$ a distância do centro da agulha até à linha paralela mais próxima e seja $\theta$ o ângulo formado entre a agulha e um das linhas paralelas, definimos então a densidade de probabilidade de x estar em $0\leq x\leq \frac{L}{2}$ como
\begin{align}
\rho(x)=\begin{cases}
\dfrac{2}{L}~&:~~ x\in\left[0, \frac{L}{2}\right]\\
\vspace{.1cm}
0~&:~x\notin \left[0, \frac{L}{2}\right]\,,
\end{cases}
\end{align}
note que aqui $x=0$ é a agulha centrada em uma tira e $x=\frac{L}{2}$ equivale à agulha com o centro entre duas tiras. A distribuição de probabilidade é uniforme, portanto a agulha possui igual 

probabilidade de cair em qualquer ponto nesse intervalo, porém não pode cair fora desse intervalo
%\begin{figure}[h!]
%	\centering
%	\includegraphics[scale=.5]{needledensity}
%\end{figure}

A densidade de probabilidade uniforme do ângulo é dada por
\begin{align}
\varrho\left(\theta\right)=\begin{cases}
\dfrac{2}{\pi}~&:\theta\in\left(0,\frac{\pi}{2}\right]\\
0~&:\theta\notin \left(0,\frac{\pi}{2}\right]\,,
\end{cases}
\end{align}
de forma semelhante, $\theta=0$ equivale a agulha caindo paralelamente às tiras e $\theta=\frac{\pi}{2}$ equivale à agulha caindo perpendicularmente. As duas variáveis aleatórias $x$ e $\theta$ são independentes, portanto, a função de probabilidade conjunta equivale ao produto de $\rho$ com $\varrho$:
\begin{align*}
P\left(x,\theta\right)=\rho\left(x\right)\varrho\left(\theta\right)\,,
\end{align*}
ou seja,
\begin{align}
\varrho\left(\theta\right)=\begin{cases}
\dfrac{4}{L\pi}~&:\left(x,\theta\right)\in D\\
0~&:\left(x,\theta\right)\notin D\,,
\end{cases}
\end{align}
aqui a região $D$ é representada por $$D=\left\{\left(x,\theta\right)\in\mathbb{R}^2;~0\leq x\leq \frac{L}{2}, 0< \theta\leq \frac{\pi}{2}\right\}\,.$$
a agulha tocará uma tira caso:
\begin{align*}
x\leq \frac{\ell}{2}\sen\left(\theta\right)\,.
\end{align*}

A partir dessa formulação temos dois casos: Uma agulha curta e uma longa. Para o primeiro, integramos $P\left(x,\theta\right)$ sobre toda região\footnote{Lembre-se que $D=\left\{\left(x,\theta\right)\in\mathbb{R}^2;~0\leq x\leq \frac{L}{2}, 0< \theta\leq \frac{\pi}{2}\right\}\,.$} para encontrar a probabilidade da agulha cruzar uma tira:
\begin{align}\label{eq5}
\mathcal{P}=\iint\limits_{D}P\left(x,\theta\right)dA=\int\limits_{0}^{\frac{\pi}{2}}\left(\int\limits_{0}^{\frac{\ell}{2}\sen\left(\theta\right)}\frac{4}{\pi L}dx\right)d\theta=\frac{2\ell}{\pi L}\,.
\end{align}

Agora para o segundo caso, a agulha é maior que a tira, ou seja $\ell > L$, devemos para esse caso supor que exista algum mínimo $m\left(\theta\right)$ entre $\frac{\ell}{2}\sen\left(\theta\right)$ e $\frac{L}{2}$, para esse caso a probabilidade é dada por
\begin{align}
\mathcal{P}=\int\limits_{0}^{\frac{\pi}{2}}\left(\int\limits_{0}^{m\left(\theta\right)}\frac{4}{\pi L}dx\right)d\theta=\frac{2\ell}{\pi L}-\frac{2}{\pi L}\left[\sqrt{\ell^2-L^2}+L \arcsen\left(\frac{L}{\ell}\right)\right]+1
\end{align}
ou equivalentemente
\begin{align}
\mathcal{P}=\frac{2}{\pi}\arccos\left(\frac{L}{\ell}\right)+\frac{2\ell}{\pi L}\left[1-\sqrt{1-\left(\frac{L}{\ell}\right)^2}\right]\,,
\end{align}



na segunda expressão, o primeiro termo representa a probabilidade do ângulo da agulha de forma que ela atravessará sempre ao menos uma tira. O termo da direita representa a probabilidade da agulha cair formando um ângulo que dependa da posição e que atravesse a tira.

Também perceba que sempre quando $\theta$ possuir um valor tal que $\sen\left(\theta\right)\leq L$, a probabilidade de atravessar a tira é a mesma que o caso da agulha curta, no entanto se $L\sen\left(\theta\right)> L$ a probabilidade será constante e igual a 1.

\paragraph{Resultados}

Para estimar $\pi$ devemos reorganizar a eq. \ref{eq5} como
\begin{align*}
\pi=\frac{2\ell}{L\mathcal{P}}\,,
\end{align*}
ou seja, se conduzirmos um experimento para estimar $\mathcal{P}$ conseguiremos uma estimativa para $\pi$. Suponha que em um experimento jogamos $n$ agulhas e verificamos que $m$ dessas agulhas estão cruzando as tiras, então uma aproximação de $\mathcal{P}$ é dada por $$\mathcal{P}=\frac{m}{n}$$ e obtemos daí a fórmula \begin{align}
\pi\approx\frac{2\ell n}{mL}\,.
\end{align}


Em 1901, Mario Lazzarini, um matemático italiano realizou o experimento de Buffon, jogando uma agulha 3408 vezes, ele obteve uma aproximação razoável para $pi$ cuja precisão resultou em 6 casas decimais de $\pi$.
\bigbreak
	
	
	
\end{document}